Disagreements and conflict are vital for driving scholarly progress, social and scientific alike. In research, we often identify gaps in others' and our own work, to present new ideas that remedy them. Disagreements are often small in nature: We disagree on methods rather than the research programme itself. In this talk, we discuss a disagreement of a different nature: namely one in which the substance of the disagreement is the existence of the task itself. We reflect on the experience of the conflict, how it was resolved, and what outcomes it has had.
\\
In particular, Liwei will share her current interdisciplinary research journey on AI + humanity sparked by the Delphi experience. She will introduce Value Kaleidoscope—a novel computational system aiming to model potentially conflicting, pluralistic human values interwoven in human decision-making. Finally, she will talk about an exciting co-evolution opportunity unfolding between frontier AI technology and humanity fields.
\\
Zeerak will go over ongoing work that considers the foundations and limits of machine learning and NLP with regard to ethically appropriate work. Specifically, they will discuss the use of the distributional hypothesis, and what particular visions of our societies it offers, and how machine learning seeks to construct our future in the vision of the past.